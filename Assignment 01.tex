\documentclass[titlepage]{article}
\usepackage{graphicx}
\graphicspath{ {./images/} }
\usepackage{kotex}
\usepackage{wrapfig}
\usepackage[export]{adjustbox}
\usepackage{subcaption}
\usepackage{hyperref}



\begin{document}

%표지
\begin{titlepage}

\title{Human Interface Imaging\\Assgignment01}
\author{20112096\\최준영}
\maketitle
\end{titlepage}

%목차
\tableofcontents
 
    
%본문
\newpage
\section{ github 시작하기}
	
		\subsection{ github의 용도}
   		\begin{figure}[h]
		\includegraphics[width=0.5\textwidth, center]{image1}
		\caption{github}
		\label{fig:figure1}
		\end{figure}

\par
\noindent github는 온라인 상의 원격 저장소와, 나의 컴퓨터의 로컬 저장소를 연결해 작업물의 수정, 업로드, 다운로드 등 작업물의 전반적 관리/공유 과정을 간편하게 도와준다.
\par
\noindent github를 이용하기 위해서는 먼저, github.com에 접속해 나의 원격저장소(repository)를 생성해야한다.\\

		\subsection{ github에 repository 생성하기}
			
			github에 저장소를 생성하기 위해서는 먼저 계정을 생성해야한다. 간단한 정보만을 입력하면
github 계정을 만들 수 있다.
New Repository를 누르면 원격 저장소를 생성할 수 있다. 이 원격 저장소는 단순히 자료를 업로드하는 것
뿐만 아니라 자료를 배포하거나 업데이트하고 또한 다른 이에게 공개하여 협업을 진행할 수도 있다.
github에 원격저장소를 만들면서, Readme.md 파일을 생성할 지 말 지를 선택할 수 있다.
\par
\noindent 이번 과제에서는 git의 사용법을 익히는 것이 주된 내용이므로, github 웹에서 생성하지 않고 
git을 통해 Readme.md를 로컬에 생성, 원격 저장소로 업로드할 것이다.\\
		\begin{figure}[h]
 
		\begin{subfigure}{0.5\textwidth}
		\includegraphics[width=0.9\linewidth, height=5cm]{image3} 
		\end{subfigure}
		\begin{subfigure}{0.5\textwidth}
		\includegraphics[width=0.9\linewidth, height=5cm]{image4}
		
	
		\end{subfigure}
 
		\caption{Repo생성과 Readme.md 생성 해제}
		\label{fig:figure2}
		\end{figure}
		\\
	
\section{git의 사용}
	
		\subsection{ git이란?}
		git CLI(Command Line Interface)는 나의 로컬 저장소를 관리하는 용도로 사용하는 소프트웨어이다.
Windows의 CMD처럼 명령어에 따라 여러가지 기능을 수행한다.
\par
\noindent git은 Windows, Mac OS X, Linux/Unix 운영체제에 대해 지원하고 있으므로, 
자신의 컴퓨터 운영체제에 맞는 git을 설치하면 된다.\\
		\subsection{git의 작업 흐름}
	
		git을 통해 로컬 저장소의 공간을 git에 의해 관리받도록 지정하고,
그 공간의 작업물을 관리하며, 원격저장소로 push할 수 있다. 
\par
\noindent git은 다음의 작업 흐름을 통해 파일을 관리한다.\\
		
		\begin{enumerate}
			\item 작업디렉토리(Working Directory) : 실제 내가 작업하는, 원격저장소와 연동할 파일들이 저장된 공간이다.
			\item 인덱스(Index): 작업디렉토리에서 변경된 파일은 먼저 인덱스에 추가되어야한다. 하지만, 인덱스를
붙인 것일 뿐, 작업에 대한 설명이 전혀 없으므로 아직 github에 업로드될 준비가 된 상태는 아니다.
			\item Head: github는 매 작업마다 commit을 부여할 수 있다. 이 과정을 거치면 Head에 반영되어 
원격 저장소에 업로드될 준비가 끝난 것이다.
		\end{enumerate}
		\newpage
		\subsection{ git 시작하기- 사용자 정보 설정과 설정 확인}
		git 설치를 끝내면 Git Bash를 통해 본격적으로 git을 활용할 수 있다. \\
		\begin{figure}[h]
		\includegraphics[width=0.5\textwidth, center]{image5}
		\caption{gitBash 실행 화면}
		\label{fig:figure3}
		\end{figure}
		\\
		git을 설치하고 나면 가장 먼저 사용자 이름과 이메일 주소를 설정해야 한다. 이때 설정한 정보를 바탕으로 추후 commit을 진행한다.\\

\par
\noindent\$ git config --global user.name "Joonyoung Choi"\\
\$ git config --global user.email "mydream757@gmail.com"
\\\\
 만약 프로젝트마다 다른 정보를 사용하고 싶으면 --global 옵션을 빼고 명령을 실행하면 된다.\\
 		\begin{figure}[h]
 
		\begin{subfigure}{0.5\textwidth}
		\includegraphics[width=0.9\linewidth, height=5cm]{image6} 
		\end{subfigure}
		\begin{subfigure}{0.5\textwidth}
		\includegraphics[width=0.9\linewidth, height=5cm]{image8}
		
	
		\end{subfigure}
 
		\caption{사용자 설정과 확인}
		\label{fig:figure4}
		\end{figure}
	\par
\noindent git config {key} 명령으로 Git이 특정 Key에 대해 어떤 값을 사용하는지 확인할 수 있다.\\

ex) \$ git config user.name\\

\begin{figure}[h]
		\includegraphics[width=0.5\textwidth, center]{image9}
		\caption{사용자 이름만 확인}
		\label{fig:figure5}
		\end{figure}
		\subsection{ 저장소 만들기}
		Git 저장소를 만드는 방법은 두 가지다. 기존 프로젝트를 Git 저장소로 만드는 방법이 있고 다른 서버에 있는 저장소를 Clone하는 방법이 있다. 

\par
\noindent 기존 프로젝트를 저장소로 만들어 Git으로 관리하고 싶을 때, 프로젝트의 디렉토리로 이동해서 아래과 같은 명령을 실행한다.\\

\$ git init\\
\begin{figure}[h]
 
		\begin{subfigure}{0.5\textwidth}
		\includegraphics[width=0.5\linewidth]{image9} 
		\end{subfigure}
		\begin{subfigure}{0.5\textwidth}
		\includegraphics[width=0.5\linewidth]{image10}
		
	
		\end{subfigure}
 
		\caption{저장소로 만들기}
		\label{fig:figure6}
		\end{figure}
		\par
\noindent 또는 디렉토리로 접근이 불편한 경우, 다음과 같이 (윈도우 기준) 폴더를 우클릭해서 바로 git Bash를 통해 디렉토리를 찾을 수도 있다.\\
		\begin{figure}[h]
 
		\begin{subfigure}{0.5\textwidth}
		\includegraphics[width=0.9\linewidth, height=5cm]{image12} 
		\end{subfigure}
		\begin{subfigure}{0.5\textwidth}
		\includegraphics[width=0.9\linewidth, height=5cm]{image13}
		
	
		\end{subfigure}
 
		\caption{디렉토리 접근}
		\label{fig:figure7}
		\end{figure}
		\\
나는 Computer\_Vision폴더를 로컬 저장소로 사용할 것이기 때문에,
이전에 예시로 만든 저장소의 .git파일을 삭제할 것이다. 숨김 표시된 .git 파일을 삭제하면 git에 의해 로컬저장소로 관리되지 않는다.
\\

\section{ 로컬저장소와 원격저장소 연결하기}

		\subsection{ github의 원격저장소 Clone하기}
		다른 프로젝트에 참여하거나 기존 있던 github의 원격 저장소를 로컬에 그대로 복사하고 싶다면 clone을 사용하면 된다.
git clone [url] 명령으로 저장소를 Clone한다. \\

\$ git clone https://github.com/mydream757/Computer\_Vision.git\\
\begin{figure}[h]
 
		\begin{subfigure}{0.5\textwidth}
		\includegraphics[width=0.9\linewidth, height=5cm]{image14} 
		\end{subfigure}
		\begin{subfigure}{0.5\textwidth}
		\includegraphics[width=0.9\linewidth, height=5cm]{image15}
		
	
		\end{subfigure}
 
		\caption{Clone하기}
		\label{fig:figure8}
		\end{figure}
		\\
위의 그림을 보면, 해당 명령어를 통해 'Computer\_Vision' 이라는 원격저장소를 클론하여 같은 이름의 폴더를 생성하고 .git파일을 추가함으로써 remote까지 완료했다는 결과를 확인할 수 있다.\\
		
		\subsection{변경된 파일을 github로 Push하기}
Push를 실습하기 위해 미리 만들어둔 Readme.md 파일을 변경하고 그것을 github로 push할 것이다. 
Readme.md 파일을 수정하는 데에는 Atom 에디터를 사용하였다.\\
\begin{figure}[h]
		\includegraphics[width=0.4\textwidth, center]{image16}
		\caption{Readme.md 수정}
		\label{fig:figure9}
		\end{figure}

\par
\noindent 변경된 파일을 git을 통해서 push할 것이다. 먼저 git status를 명령하여 파일의 상태를 확인하면, 수정됨(modified)로 Readme.md 파일이 표시된다. 이것이 현재 not staged이므로 이를 staged 시켜주려면 git add를 해주어야한다.\\
\begin{figure}[h]
 
		\begin{subfigure}{0.5\textwidth}
		\includegraphics[width=0.9\linewidth, height=5cm]{image17} 
		\end{subfigure}
		\begin{subfigure}{0.5\textwidth}
		\includegraphics[width=0.9\linewidth, height=5cm]{image18}
		
	
		\end{subfigure}
 
		\caption{status 확인과 git add}
		\label{fig:figure10}
		\end{figure}
		\par
\noindent 이제 수정 내용을 github로 commit해주어야한다. git commit 명령은 vim 편집기로 이동되어 commit을 작성할 수 있고, git commit -m 명령으로 이 과정을 간소화할 수 있다. commit을 완료하면 이제 git push origin master 명령을 통해 push를 하면 다음과 같은 결과를 얻을 수 있다.\\

\begin{figure}[h]
 
		\begin{subfigure}{0.5\textwidth}
		\includegraphics[width=0.9\linewidth, height=5cm]{image19} 
		\end{subfigure}
		\begin{subfigure}{0.5\textwidth}
		\includegraphics[width=0.9\linewidth, height=5cm]{image20}
		
	
		\end{subfigure}
 
		\caption{commit과 push}
		\label{fig:figure11}
		\end{figure}
		
		\newpage
\section{ 참고자료 및 github 링크 첨부}
git개념과 사용법\\
https://git-scm.com\\\\

vim 편집기 사용\\
https://stackoverflow.com\\\\

github link\\
\href{http://www.github.com/mydream757/Computer\_Vision}{mydream757/Computer\_Vision}


\end{document}
